\section{Understanding the Research Problem}
\label{Chapter:Problem Statement}
{
    \subsection{The Problem Statement}
    {
        The main problem is to devise a solution to use a robotic manipulator with a broad edge tool to create and copy Islamic calligraphy specimens. This problem can be divided in four main questions:
        \begin{itemize}
          \item How to copy and create broad-edge scripts in the computer?
          \item How to format the data in digital format to include most of the script information?
          \item How to convert the digital data into machine data to be used with actual manipulators?
          \item How accurate and robust is the proposed solution.
        \end{itemize}

        The first task would be to study the existing digital forms of Islamic calligraphy which are mainly based on glyphs and splines that form digital computer fonts. They carry the least information to reproduce the final ink-marks of the script. They don’t carry information about the machine movement. Conventional bezier splines could also solve the problem if they could either be translated into machine data or could somehow contain machine data within in the first place. This new method then needs to be tested and characterised. This is why a simulation solution would also be needed.

    }
    \subsection{The Proposed Solution}
    {
        As we discuss in section \ref{Chapter:Twisting Splines}, the first three questions of the research can be directly answered by one innovation in the conventional bezier splines. We discuss in detail how the twisting splines work but the main idea is to include a human artist in the process. The artist is given a computer interface that mimics a broad edge tool in the same fashion similar to most vector graphics editors that work well for round tip pens. The main strength of the innovation lies in the fact that though the artist may be oblivious to what they are actually producing, they actually are also feeding direct machine data information in the spline. The output is vector data; sum of a number of continuous step functions and can be considered directly as the end effector movement data.

        Once an editor for such splines is ready, the main challenge would be the characterization and testing of the proposed technique. Section \ref{Chapter:Twisting Splines} also shares some results and benchmarks of the twisting splines.

        To verify how well the splines work with a real manipulator, a $6$ DoF robotic manipulator would be needed. Section \ref{Chapter:Drogon} discusses such a simulator written specifically to test the splines. We can use the graphical and programmatic interface of the simulator to test and quantitatively analyze the rotating bezier splines with different configurations of a robot.
    }
}
