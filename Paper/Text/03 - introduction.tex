\section{Introduction}
\label{Chapter:Introduction}
{
    Bezier curves are commonly used to define outline fonts [cite] and digital calligraphy [cite] due to their ability to accurately trace[cite] the outline of scripts written with and without a broad edge tool[cite]. These curves can be easily manipulated [cite] and rendered [cite] on a computer screen as well as they can be printed on paper. However, printing is not always a choice; in many cases, it is desired that a script is written using a conventional tool by a robotic manipulator [cite] in a similar fashion a human artist would have used in order to preserve the essence of artistic norms [cite]. Separate techniques [cite] are needed to convert output of the outline or pitch curve functions to a format that a robot can directly use to for the required tool movement. Even though many of these techniques [cite] can promise accurate tracing, none of them can promise an exact tool movement that a human would like to use. Also, once the machining data is produced the idea to manually fine tune is an intuitive one. Twisting bezier splines bridge this gap by introducing a small but important innovation in the conventional bezier spline curves. Instead of working as outline curves, they directly record the pitch line of a twisting broad-edge tool stroke along-with the twist information of the tool. Their usage is just as intuitive as the conventional bezier spline curve and the information they poses can not only be used to render the script back on the screen or a photo printer, but also be directly considered as machine data for robotic manipulators.

    In the most literal terms, beizer spline curves are sets of decimal values that define the graphical shape of certain mathematical functions [cite]. The input of these functions contains two dimensional points located in the frame of reference of the screen on which they are created and work as handles to control the shape of the curve. A user can physically relocate these curvature handles and the shape of the resulting spline will follow. The output of these mathematical functions is absolutely repeatable and can be linearly scaled to any units. Although these functions are continuous, there output can easily be discretized as closed paths made up of closely located two dimensional points with controllable resolution. This is exactly the kind of information required by most of computer graphics and printing drivers to render an output.
    
    Now, as effective as the bezier curves are for screen and paper printing, the rotating bezier splines cannot tell how a physical tool should move on a piece of paper to create the desired output. The splines are just organic shaped paths with no thickness. One way to interpret them is to consider them as a pitch line for a thick tool tip. This technique is used by plotters [cite] and some hand writing replicators [cite] to produce a written script. However, only a few fonts [cite] can be replicated by this technique. The other technique is to fill the glyph by moving the tool continuously on a path computed by algorithms such as those used by CAD tools to fill in the outline of the glyph using a round tip tool. The later can produce outputs that looks similar to broad-edge calligraphy but will still not be the same due to the visible tool paths that are not expected when using an actual broad edge tool.

    On the other hand, for a particular glyph created with a broad edge, the twisting bezier spline curves not only trace the pitch lines of all the strokes but also the twist of the tool independent of the curvature. This is done by introducing another input to the curve function we call the ``Rotation/Twist'' handle. Just like the curvature handles represent the curve function inputs responsible to define the curvature, the rotation handles represent the inputs that control the twist of the simulated broad-edge tool. Just like the conventional bezier splines, the functions of the twisting bezier splines can also be discretized and converted into a list of two dimensional points, a closed path to emulate the ink-mark of the broad edge tool, needed by the computer display drivers. This is how an artist can intuitively use the twisting curves not just to trace but also to create calligraphy that is not bound by any culture or language.
    
    The interesting part is that, logically, the twisting beizer splines are more near to the machine than they are to computer graphics driver. To compute the list of the points needed to create a filled path that represents the ink-mark, a broad edge tool is emulated to be moving on the rotating spline with the twist also controlled by the spline. In actuality, the emulated tool is replaced by an actual tool that can be mounted on a robotic manipulator. The spline directly controls the position and twist of the tip of the tool which can directly be translated into machine movement codes. The rest of the process is inverse kinematics and is already handled by the robot controller.
    
    The article is divided in $5$ sections. After the introduction, Chapter \ref{Chapter:SplineModelling} presents the working principle and mathematical model of the twisting/rotating bezier splines. In section \ref{Chapter:Performance} we discuss some performance metric and discuss the tests performed to gauge the performance of the twisting bezier splines. Section \ref{Chapter:Simulation} discusses the simulation of a robotic manipulator to verify the results. We finally conclude in Section \ref{Chapter:Conclusion}. Also see Appendix \ref{Appendix:Gregor} and \ref{Chapter:Drogon} which shows a user manual of the software tools written and used in this research, Appendix \ref{Appendix:sourceCode} which is a digital copy of the source codes, this thesis and some videos on how to reuse the tools and the code written.
} 