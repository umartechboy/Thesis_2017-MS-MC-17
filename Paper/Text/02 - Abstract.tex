\begin{abstract}
\label{Chapter:Abstract}
{
    Bezier curves can trace the outline of digital fonts and calligraphy scripts written with and without a broad-edge tool with acceptable accuracy. However, this curve data is only good enough for rendering the scripts on computer screens and photo printers; the data needs complex algorithms that can convert it into a format which is usually required by a robotic manipulator equipped with a conventional broad edge tool. This work introduces a new kind of bezier spline that can not only be manipulated and rendered on the screen as intuitively as other splines but bridges the gap between digital vector data and machining data by altogether removing the step where outline vectors or images are artificially analyzed to estimate the broad-edge tool movement on a pitch line. These rotating bezier splines are first characterised using two scripts of Islamic calligraphy to compute a performance metrics. The created scripts are then tested using visual simulations of a robotic manipulator that writes the scripts virtually on a piece of paper. It is believed that with further contribution and adoption, this research can be considered a major milestone in mechanising broad-edge calligraphy.
}
\end{abstract}
