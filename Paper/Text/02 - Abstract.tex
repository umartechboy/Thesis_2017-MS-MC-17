\begin{abstract}
\label{Chapter:Abstract}
{
    Mechanised reconstruction of Islamic architectural calligraphy can save thousands of nobel and historic scripts all over the world from going extinct. One of the fundamental problems in the job of robotic reconstruction is the absence of a tool that can bridge the gap between artists and the machine by taking input from calligraphy artists in a conventional way and can producing an output that can be used with existing industrial robots. Not just reconstruction, even in this digital age, no such method exists that lets the artists generate new broad-edge scripts that can be reproduced mechanically. This research introduces a novel innovation in the conventional bezier spline curves to use them to effectively answer the artistic requirements of copying or creating broad-edge calligraphy scripts and can directly produce data need by the industrial robots. To mathematically demonstrate the effectiveness, two famous scripts are reproduced using the new splines and a comparison is made with the originals. Additionally, to demonstrate how these splines can be used with machines, a robotic simulator is also discussed and the output analysed both qualitatively and quantitatively. It is believed that this research can be considered a major milestone in mechanising Islamic calligraphy given more attention and resources.
}
\end{abstract}
