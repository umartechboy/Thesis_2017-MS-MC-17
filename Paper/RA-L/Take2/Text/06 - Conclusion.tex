\section{Conclusion}
\label{Chapter:Conclusion}
{
    \noindent The accuracy analysis presented in this article demonstrates how rotating bezier splines closely mimic the ink mark of broad edge writing tools thus creating accurate calligraphy scripts. They can be traced on photos of existing calligraphy scripts while also involving the artists to better preserve the artistic spirit. Their resemblance with normal bezier splines makes the learning curve lean for the people who are not accustomed with engineering software and tools. Once traced, they can directly generate machine data for robotic reconstruction and be reproduce the original script on flat and uneven surfaces. Tracing accuracy, ease of use and the ability to generate machine data makes them a strong candidate, not just for Islamic calligraphy but also other broad-edge scripts, to bridge the gap between machines and artists to reproduce and innovate in this field.

    Furthermore, rotating bezier splines can be involved in optical character recognition of Islamic scripts. Since they can be regarded as mathematical curves of the actual ink-marks, a modified curve-fitting technique can be developed to find the perfect curve to match a particular calligraphy script. At the least, the open-source proof-of-the-concept editor presented in this research can be mixed with such a curve fitting tool to help fine tune the manual tracing process.
} 