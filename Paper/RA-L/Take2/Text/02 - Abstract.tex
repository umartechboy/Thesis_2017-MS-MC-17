\begin{abstract}
\label{Chapter:Abstract}
{
    \noindent Automated reconstruction of Islamic architectural calligraphy can save thousands of noble and historic scripts all over the world from going extinct. One of the fundamental problems in such reconstruction tasks in the absence of a tool that can bridge the gap between artists and the machines (such as robotic arms) by taking input from conventional calligraphy format and transforming it into a robot program. Not just the reconstruction, even in this digital age, no such method exists that lets the artists generate new broad-edge scripts that can be reproduced mechanically. In this context, the research presented herein introduces a novel innovation in the conventional bezier spline curves to use them to effectively answer the artistic requirements of copying or creating broad-edge calligraphy scripts and can directly produce data needed by the industrial robots. To demonstrate the effectiveness of our approach, two famous scripts are reproduced using the new splines and a comparison is made with the originals. Additionally, to establish how these splines can be used with machines, a robotic simulator is also discussed and the output is analyzed both qualitatively and quantitatively. In addition to the mechanized reconstruction of Islamic calligraphy, the presented approach can also be used for generating calligraphy for virtual 3D models, e.g. in Metaverse, 3D documentation of non-reachable and ruined buildings, and even OCR of Islamic calligraphy scripts.
}
\end{abstract}
