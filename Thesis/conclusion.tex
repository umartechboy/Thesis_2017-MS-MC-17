\section{Conclusion} \label{Chapter:Conclusion}
{
    The twisting Bezier spline curves very closely mimic the ink mark of broad edge tools thus creating very accurate calligraphy scripts. With most of broad edge scripts, on an average, they give more than 95\% coverage of the reference image an less than 5\% overdraw. This has now been verified by comparing them with original calligraphy specimens as well as after simulating the output of a robotic manipulator.
    
    With tools created as a by-product of this research, an artist can not only trace existing calligraphy scripts, but also create and modify new ones with a very lean learning curve. The ease of use of the tools created for the tast assures that the focus of the artist is more on the art itself than the caveats of the software solution.
    
    Where the accuracy of the splines has been characterised and ease of use has been demonstrated, there still are a lot of unexplored areas that require more research. As discussed in section \ref{ExplorationPoints1}, one can pack the tool inclination and normal pressure information into the rotating splines. One can also choose to implement other kinds of manipulators and actuators to test the performance of the splines in the simulator. Last but not the least, since the simulator can emulate a pseudo robot, it can also be modified to be used as a live controller for a real robotic manipulator.
    
    While this thesis puts together the results of some crucial tests to establish the usefulness of rotating splines, the final verdict will still be given by the community that takes the work forward in more scenarios and conditions.
}
\clearpage 