\begin{abstract}
{
    Mechanising Islamic calligraphy is a challenging job that demands research both in the process of digitization of existing art and the creation of new scripts. This research first yields a novel innovation in the conventional bezier spline curves to use them to effectively answer the artistic requirements of copying or creating broad-edge calligraphy scripts and then mathematically compares the output with the original specimens. Since these twisting spline curves also claim to bridge the gap between digital script data and a robotic manipulator that needs machine data to start moving, a robotic simulator is also discussed. The robustness of the simulator and its efficiency with the twisting bezier splines is also discussed. The research also proposes a novel method to represent, design and solve robotic manipulators. It is believed that this research can be considered a major milestone in mechanising Islamic calligraphy given more attention and resources.
}
\end{abstract}
