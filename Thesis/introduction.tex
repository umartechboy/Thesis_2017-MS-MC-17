\section{Introduction}
{
    Islamic calligraphy is an art having a history that dates back to the seventh century \cite{bib01, bib02}. It has witnessed many evolutionary stages \cite{bib02, bib03} and has been used by artists speaking several different languages \cite{bib04} and sharing uncommon biographies \cite{bib05,bib06,bib07,bib08}. Unfortunately though, the industrial age and the advent of technology has not spared this beautiful art when it claims to provide better alternatives for almost everything related to human beings. Discovery of new facets of calligraphy aside, with the prevalence of modern technologies and resulting lack of expertise in this domain, the very existence of Islamic calligraphy now faces a serious threat. Public buildings and infrastructure that once used to be a showcase for the most laudable artists of the time have turned in-to museums; awaiting to be wiped away slowly with each round of the monsoon and every splash of the ocean’s waves.

    Potentially, we can use robotic dexterity to help us in this domain. Industrial robots have already been used outside the industry to do unorthodox tasks \cite{bib09, bib10,bib11,bib12} and they can surely uplift this art as well. At the very least, they can be employed in restoration and replication of existing calligraphy work \cite{bib13}. In other words, they can be used as printers, or rather one may say, “painters” that give an extra hand to the calligraphy artists to open up a new dimension of art that can not only revamp the existing calligraphy sites but also create new ones.

     Mechanized/robotic drawing of the Islamic calligraphy scripts requires not just the ink-mark information but also the information about the tool movement \cite{bib03}. Specially, using a flexible broad edge brush instead of a solid round tip pen and all that to draw on un-even surfaces, makes the job extremely special indeed. A robot needs to take special care about the orientation and downwards force of the tool as well.

    In a nutshell, the main problem can be divided in two major sections. First, transforming the printed scripts into machine data and second, recreating the scripts using a robotic end effector.
    
    To solve the first part, instead of doing image processing, we propose a new way of transforming the existing scripts into machine data; the “Rotating/twisting Bezier Spline Curves”, or simply, ``Twisting Bezier Splines''. The idea is to bring real artists in the process. For the new scheme to be fully tested, a fully featured graphical spline editor and analyzer would also be needed. The tool was tested and tuned with the help of multiple real-world calligraphers.

    To answer the second part of the problem, we discuss the working of a robotic simulator written specifically for the research. The usage, working principle and some benchmarks of the simulator are also shared.
    
    The thesis is divided in seven sections. After the introduction, section \ref{Chapter:Problem Statement} states the problem and proposes a solution and discusses the challenges that need to be overcome. Discussing the detail of the proposed solutions, section \ref{Chapter:Twisting Splines} discusses the principal, working and testing of twisting bezier splines. In section \ref{Chapter:RoboticManipulator} we discuss the modelling of the robotic manipulator used in the research. Section \ref{Chapter:Drogon} and \ref{Chapter:Gregor} discuss the software tools written specifically for the research. We finally conclude in Section \ref{Chapter:Conclusion}.
} 